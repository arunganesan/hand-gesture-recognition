\documentclass{article} % For LaTeX2e

\usepackage{nips11submit_e,times}

\usepackage{epsfig,endnotes}
\usepackage{graphicx, amsfonts, amsmath, amsthm, wrapfig, color}
\usepackage{hyperref}
\usepackage[parfill]{parskip}
\usepackage{algorithm}
%\usepackage{algorithmic}
\usepackage{algpseudocode}
\usepackage{subfig}
\usepackage{float}
\usepackage{fullpage}
 

\newcommand{\includeimage}[2] {\fbox{ \includegraphics[width=#1]{#2}}}
\newcommand{\todo}[1]{{\textcolor{red}{[[TODO: #1]]}}}
\title{A Real-time Hand Gesture Recognition System}


\author{
Arun Ganesan \\
Department of Electrical Engineering Computer Science\\
The University of Michigan\\
Ann Arbor, MI 48109-2121 \\
\texttt{caoxiezh@umich.edu} \\
\And
Caoxie (Michael) Zhang \\
}

% The \author macro works with any number of authors. There are two commands
% used to separate the names and addresses of multiple authors: \And and \AND.
%
% Using \And between authors leaves it to \LaTeX{} to determine where to break
% the lines. Using \AND forces a linebreak at that point. So, if \LaTeX{}
% puts 3 of 4 authors names on the first line, and the last on the second
% line, try using \AND instead of \And before the third author name.

\newcommand{\fix}{\marginpar{FIX}}
\newcommand{\new}{\marginpar{NEW}}

%\nipsfinalcopy % Uncomment for camera-ready version

\begin{document}


\maketitle

\begin{abstract}
Hand gesture recognition can be used in many applications such as interactive 
data analysis or American Sign Language detection. Current systems are either expensive, unable to run in real time, or require the user wear devices such as custom gloves. We propose an inexpensive solution for predicting hand gestures in real time that uses Microsoft's Kinect camera. Our system involves training a random forest classifier with a color glove, and then predicting at a pixel level a naked hand. Our system predicts all pixels at about 10 fps, and is resilient to environment differences in prediction. We also conduct extensive experiment studying the random forest classifier and reveal some interesting properties.
\end{abstract}

\section{Introduction}
Natural user interface (NUI) is a new way for human to interactive with machines. Among numerous NUIs which include multi-touch screen, eye tracking and many others, hand gesture seems to be one promising candidate. In this paper, we design and evaluate a novel hand gesture recognition system to demonstrate that we are close to an actual production-level system. The reader should be noticed that we do not claim hand gesture is THE future user interface, and in fact there are some limitations for using hand gestures such as users may feel fatigues (in the movie Minority Report, Tom Cruise has to take breaks many time due to fatigue). Similar to many other new UI systems, we propose our system as one (interesting) way to interact with the computer. We do not claim that the system would replace mouse and keyboard and we leave the usability problem to future research as building hand gesture recognition system alone is quite challenging. 


\subsection{Design Goals}
Our system is designed to maximize user experience. Moreover, our system differs from other existing systems in the following ways.

\textbf{Just hands.} The users do not need to attach any additional physical objects to use our systems. They just need to show up their hands. Many existing system such as [SixSense, MIT Minority Report, MIT color glove] require users to wear gloves or markers  for the RGB camera to capture. We eliminate this since the user should not do anything more than showing their hands. 
  
\textbf{Real-time.} Our system should run smoothly on a modern machine with a graphical card not just on high-end machines. The system should also recognize hand gesture at a high frame rate. Our desired frame rate 30Hz, although the current version has around 5Hz. In our design and implementation, one driving goal is to squeeze every milliseconds as possible. 

\textbf{No calibration.} Our system should not require a new user to do anything to calibrate the system to be used to the user. 

\textbf{Robust and Accurate.} Our system should have an accurate estimation of where the users' hands are and what gestures they use with low false positive. Moreover, the system should be insensitive to various background, user's location, camera position and other noise. 
  
\textbf{Arbitrary gestures.} Our system should be able to easily incorporate new types of gestures that any developers would like to add. By training new gestures, the system can recognize arbitrary gestures, for example the American sign language. 

\subsection{Main Ideas}
Our system would not be possible without the use of Microsoft Kinect for PC, which we are probably among the first to obtain it in February 2012. Kinect is a multi-purpose sensor including RGB camera, depth camera and audio sensor. The Kinect SDK has offered skeletal recognition, which is however far away from recognizing hand gesture. The SDK also provides raw pixels for the RGB image and depth image at a maximum frame rate 30Hz. We use the depth image for gesture recognition and both RGB and depth image for generating training samples. The depth image is the key factor that distinguishes our system from most existing systems that use RGB camera. The advantage of the depth image is that it offers an addition dimension, i.e. depth of each pixel that is not present in the RGB image. An illustrating example would be an object and its background has similar color but the depth of the two is drastically different. 

Our system adopts a data-driven approach: machine learning as opposed to hand-crafted rule-based systems. The adoption of machine learning transfers the human intelligent efforts from design rules/algorithms to design informative features. With the features on labeled data, machine learning allows computers to learn the rule/algorithm automatically. The key advantages of using machine learning in our system    
are (1) easy to incorporate developer-defined gestures: developers just need to feed the system with the gesture images to be trained rather than deriving new algorithms (2) robust to various environments such as camera position, background, various size of the hands: developers just need to generate the gestures on various environments without worrying about anything more. 

In a high-level overview, the system is separated into two parts: training and real-time prediction. Only the real-time prediction component is seen by the end users. In the training component, we use color gloves to generate massive labeled data of the depth image. Random forest is trained to achieve both real-time performance and high accuracy. In the real-time prediction component, each pixel in the depth image is predicted the type of the gesture using GPU and then the prediction outputs are pooled to propose the final position and type of a gesture. Notice that we do not use any temporal or kinetics information as the current simple design suffices for the hand gesture tracking.   [Need to elaborate more?] 



\subsection{Contributions}

We summarize our contributions as follows:

\textbf{A system for real-time hand gesture recognition.} We design and implement a complete real-time hand gesture recognition system based on machine learning. The system can be used as API for other applications to read the gesture in real-time.

\textbf{An computational insight about random forest and support vector machine (SVM).} To the best of our knowledge, there seems to be no literature in comparing  SVM and random forest from a computational perspective. We provide an in-depth comparison in the angle of performance rather than merely predictability as done in most machine learning literature. 

\textbf{Extensive experimental evaluations of the system.} We conduct extensive experiments evaluation on the effectiveness of the machine learning approach, i.e., random forest we use in a large space of parameters. Interesting results reveals a deeper understanding of random forest. 

\subsection{Related Work [TODO]}

There are two main techniques in hand gesture recognition - appearance based and model based. These are akin to probabilistic models of classification and generative models of classification respectively. \textbf{Appearance based} approaches read the pixels from the camera and build classifiers to label that as belonging to a finite set of classes. The main limitation of this technique is that the set of labels is finite and fixed ahead of time. The advantage of appearance based approaches is their implementation is often extremely fast and therefore suited for situations where live classification is important. Examples of appearance based techniques can be found in \cite{shotton2011, wang2009}. \textbf{Model based} approaches start with a set of hypotheses of the final classification based on rules of the object being classified. For instance, in the case of hand gestures a hypothesis can be a particular orientation of the joints.  An advantage here is that the hypotheses can be generated from an infinite space of possible classifications. The main disadvantage of model-based techniques is that they are often computationally expensive. In addition, model-based approaches tend to be very complex. An example of a model based technique can be found in \cite{oikonomidis2011}.

\section{System Overview}

From the user point of view, our system is divided into two components: developers mode for training gestures and end-user mode for real-time prediction. In a high-level overview, developers using the developer mode supply the system with labeled data by wearing color gloves. Hence the developers hand the labeled image for the system to train. In the end-user mode, the users offer the raw image by showing up their hands, and the system will make predictions on every pixel using GPU and then pool every prediction to get an estimation of the gesture. Between the two modes, they share a component: feature extraction, which obtains the features for each pixel in the depth image. The implementation of feature extraction in the two modes are however different one is using CPU in an off-line fashion and the other is using GPU (end-user mode) in an on-demand fashion. 


%There are three main stages to our system. First is acquiring training samples using a colored glove. Second is training a classifier on the samples. Third is live prediction and pooling. 

%The approach is pixel-level classification, and future pooling. The motivation for pixel-level classification is the potential of parallelizing the task and using the GPU to achieve real-time prediction. 

\begin{figure}
\centering
	\includegraphics[width=0.5\textwidth]{fig/SystemArchitecture.pdf}
	\caption{An overview of the system architecture}
\label{fig: architecture}
\end{figure}

\subsection{The Kinect Sensor}
The Kinect sensor provides both raw depth image and color image at a maximum frame rate of 30Hz and with a resolution 640 times 480. Each pixel in the depth image represent the distance from the object to the camera. The error of the depth can be with several millimeters[TO VERIFY]. However, objects can have a shadow where some parts near the object do not have depth value at all.   

\subsection{Outline}

In the following, we go through feature extraction and per-pixel classification in Sec. X, generating training samples in Sec. Y, pooling in Sec. Z, implementation in Sec. YY, our experience in Sec.Zz, experiment in Sec. XX and conclude in sec. XX.

\section{Per-pixel Classification}

\subsection{Acquiring training samples}
Color glove approach. To simplify the data acquisition step, we used cropping, depth thresholding, and a single colored glove. We used 4 gestures, giving a total of 5 classes including the background. We collected 400 training samples evenly spread across the gestures.  

\subsection{Training classifier}
We trained the classifier on different subsets of the 400 training samples to study the effect of the training sample size. Also, we set aside 50 samples for testing purposes. Because our prediction is at the pixel level, we sample 1000 pixels from each of the training image, trying to get a good balance between the background and the gesture.

We used a depth-invariant feature vector for each pixel in our training sample. Each feature in the feature vector is the difference in the depth value of two offset pixels from the current pixel. We experimented with different number of such offset pairs for the feature vector.

We trained two classifiers - an SVM classifier for baseline testing, and a random forest classifier with multiple configurations. The random forest classifier can have multiple trees. We explored the effect of different number of trees in the forest.

\subsection{Prediction}
We aimed for real-time prediction of the hand gesture. The trained model from the classifier is used for per-pixel classification. That is, each pixel is classified as belonging to one of the four gestures or to the background. From here we have to first identify the likely gesture, and then find the location of that gesture, a process we called `pooling'. 

\subsubsection{GPU} Running the prediction algorithm using the CPU proved to be very slow. For a $640\times480$ image, our system took 2.5 minutes to classify all the pixels. Given that this problem is highly parallelizable, we turned to the GPU. Re-implementing the prediction algorithm with the GPU reduced the prediction time from 2.5 minutes to 400 milliseconds - a 99.7\% speedup!


\subsection{GPU performance enhancements}

\section{Generating training samples}
\label{sec: generating_training}

In this section we present our method for rapidly generating labeled training samples.

To rapidly generate training samples, we use a color glove that has a different color for each area of interest. We can present the glove to the camera and extract regions of a specific color using \textbf{RGB camer} and label the \textbf{corresponding depth pixel} appropriately. Examples of gloves we used in our experiments are shown in figure~\ref{fig:gloves}.

\begin{figure}
\begin{center}
\includegraphics[width=0.23 \textwidth]{fig/blueglove.png}
\includegraphics[width=0.23 \textwidth]{fig/colorglove.png}
\end{center}
\caption{Two differently colored gloves we used in rapidly generating labeled training samples. For the multicolored glove, we had to ensure the colors were different enough to be easily recognized by the RGB camera.}
\label{fig:gloves}
\end{figure}

\begin{figure}
\begin{center}

\includegraphics[width=0.11 \textwidth]{fig/gesture1.png}
\includegraphics[width=0.11 \textwidth]{fig/gesture2.png}
\includegraphics[width=0.11 \textwidth]{fig/gesture3.png}
\includegraphics[width=0.11 \textwidth]{fig/gesture4.png}

%\includegraphics[width=0.23 \textwidth]{fig/gesture1.png}
%\includegraphics[width=0.23 \textwidth]{fig/gesture2.png}
%\\
%\includegraphics[width=0.23 \textwidth]{fig/gesture3.png}
%\includegraphics[width=0.23 \textwidth]{fig/gesture4.png}
\end{center}
\caption{An example of four gestures. Each gesture represent a label of every pixel that belongs to the glove.  
%seen above were chosen because of their potential applicability to data visualization interaction systems. For example, the first figure may be used to instruct the application to translate the visualization along with the hand. The second image could lock the image in place while the other hand forms gestures.
}
\label{fig:gestures}
\end{figure}


We devised two different techniques of labeling each pixel. The first technique, shown in the left image of Figure~\ref{fig:gloves}, is to differentiate all pixels in the hand from the background. When training the system, we train one gesture at a time and classify a pixel on the hand as belonging to that gesture and classify any pixel not on the hand as part of the background. We trained four gestures using this approach yielding a total of five labels. An example of the gestures can be seen in Figure~\ref{fig:gestures}. The second technique is shown in the right image of Figure~\ref{fig:gloves}. In this technique each finger is labeled separately. Therefore each pixel can have a total of seven labels - six for the hand, and one for the background.

We also use other techniques to simplify labeling such as cropping and ignoring far away pixels via the depth information.

\section{Pooling}
\begin{itemize}
\item mean versus median -- mean is prone to outliers messing up the location. median is more resistant to that.
\item majority gesture -- if the hand is far away, this leads to the noisy labels being 
\item k-meniods -- can be used to support multiple hands at once. and can be used to filter out the noise.
\end{itemize}


\section{Implementation}
* Use Kinect SDK to map the color pixel to depth pixel
* Color gloves with fingers colored is not trained due the low resolution of the Kinect depth camera.
* OpenCL
* GPU has lot cache hit. It's limited by memory bandwidth since for the processor unit to get objects the GPU memory will take hundreds of operations
* Virtual wall
* Show actual system picture here


\subsection{Refinements}
* Changing double to float saves a lot of space
* Use EC2

\section{Experimental results}
% Bigger picture
There are many parameters involved in capturing the training images, extracting features, and training the random forest classifier. To study the contribution of these parameters, we systematically varied them and studied the accuracy of the resulting classifier. To study the accuracy, we set aside a collection of images as the test set and trained on a different set of images. We didn't use cross-validation due to the time constraints in training a random forest; it would have taken considerably longer time to test the cross validation accuracy and would  have prevented us from running as many tests as we did.

% Parameters tested
\subsection{Experiments}
We collected images of four different gestures shown in figure~\ref{fig:gestures}.

We varied three parameters in training the models. We varied the number of trees in the forest from two trees to five trees; 1000, 2000 and 3000 features; and different number of training images starting at 10, 50, 100 up to 350 increments of 50. We also set aside 50 images as the test set. For each of these 96 configurations, we trained a random forest model and computed the test set accuracy.

The features represent randomly selected offsets from the point of prediction. The offsets are radially sampled up to a maximum radius. We varied the radius of the sampling and generated feature files. We trained models with 2000 features, three trees, and different number of training images for radii set to 20, 40, 60, 100 and 200. \todo{WHAT UNIT IS THIS???}.

Based on the results, we varied the parameters more in interesting directions: (1) We studied the effect of training a forest with just one tree and ten trees, (2) we trained with 689 training images, and (3) we tried smaller radii for the feature offsets. 

In addition to varying the parameters, we conducted two more experiments. First, due to the nature of the random forest, the trained model is non-deterministic and may yield a different accuracy rate for the same training set of images. In order to study this randomness, we trained ten models on the the same training set with 1000 features, 300 training images, and one tree. Second, we studied the effect of pruning on the accuracy of prediction. To study this, we pruned the forests trained with 2000 features and three trees for different number of training images. Then we calculated the test set accuracy of the pruned models.

%Michael: I mentioned that in the Implementation section
%We used the ALGLIB's random forest implementation for training and prediction \cite{alglib}. The experiments were run on high memory EC2 instances from Amazon, and a 48 GB server. \todo{Do we even have to mention this? Not very interesting.}

\subsection{Results}
\begin{itemize}
	\item \# of trees makes no difference! But kind of does in the practice. We think it has to do with the difference in background
	\item 2000 features does better than 3000 which does better than 1000.
	\item Increasing the training size makes the largest difference.
	\item Decreasing the range leads to more accurate models faster. But as we increase the training size, it doesn't really matter.
	\item The accuracy is roughly the same across multiple randomly trained models
	\item Pruning doesn't affect accuracy, which is great news for us.
	\item The test accuracy doesn't seem to translate over the actual real-time prediction accuracy. This may be the different background.
\end{itemize}

So we built a wall.

\section{Experience}
\cutsection
\label{sec: experience}
We learned several lessons in building a practical machine learning system.

\cutequation
\textbf{Late-optimization.} Although our system is very time-sensitive, we found that it is not necessary to do optimization on one component while other components are not ready. We found it is best to develop quickly and do profiling to discover the bottleneck and optimize it. This saves us a lot of time in making unnecessary optimization.   

\cutequation
\textbf{Using pipeline rather than events.} In the testing of our system, we have incurred many concurrency bugs due to the event-driven programming framework. We solve this by not using locks but adding the processing unit to a pipeline, thus leaving us free concurrency bugs.

\cutequation
\textbf{Test accuracy is not enough.} We found that using the test accuracy is not a good evaluation of the actual system. Sometimes the system performs very well in the test data set, but poorly in practice. The reason behind this is that the environment we evaluate changes all the time: camera position, backgrounds, people's clothes, and etc. Therefore, we decide to choose some important parameters based on our experience in the actual environment instead of mere test accuracy. For example, although the experiments indicated the number of trees made no difference, we found to this to not be the case in actual tests with varying backgrounds. Therefore, we used a model trained with multiple trees rather than one.
\cutequation

\textbf{Bundle the model with feature extraction.} In our system, we separated feature extraction with the model. This turned out to be error-prone. It would happen sometimes that the trained model has the wrong feature extraction (e.g., the offset pairs are not correct), and things could get even worse when the real-time prediction system mistakenly uses the wrong feature extraction method. Our immediate solution is to be extremely careful to this. However, if we were to build the system again we would have  bundled the model with feature extraction to avoid human errors.

\cutsubsection
\subsection{Limitations}
\cutsubsection
Our current system is not without limitations. First, the real-time prediction component cannot achieve a frame rate of 30Hz but only 10Hz on average. Second, the random forest model takes a long time to train, especially with many training samples. Often times, training required around 24 hours or more. Third, the system must be retrained with more training samples every time the user wants to add a new gesture. 


\section{Related Work \todo{Need more}}

There are two main techniques in hand gesture recognition - appearance based and model based. These are akin to probabilistic models of classification and generative models of classification respectively. \textbf{Appearance based} approaches read the pixels from the camera and build classifiers to label that as belonging to a finite set of classes. The main limitation of this technique is that the set of labels is finite and fixed ahead of time. The advantage of appearance based approaches is their implementation is often extremely fast and therefore suited for situations where live classification is important. Examples of appearance based techniques can be found in \cite{shotton2011, wang2009}. \textbf{Model based} approaches start with a set of hypotheses of the final classification based on rules of the object being classified. For instance, in the case of hand gestures a hypothesis can be a particular orientation of the joints.  An advantage here is that the hypotheses can be generated from an infinite space of possible classifications. The main disadvantage of model-based techniques is that they are often computationally expensive. In addition, model-based approaches tend to be very complex. An example of a model based technique can be found in \cite{oikonomidis2011}.

\section{Conclusion}
\label{sec: conclusion}
\cutsection
% Summarize
We presented a real-time hand gesture recognition system built on top of Microsoft's Kinect SDK using their depth range camera. After several optimization, including re-implementing ALGLIB's decision forest prediction algorithm to use the GPU architecture, we were able to achieve real-time prediction at 25 fps. We extensively studied the random forest classifier by varying the multitude of hyperparameters. We discovered that (1) increasing the number of training samples had the biggest impact in the test set accuracy, (2) changing the number of trees had no noticeable effect, and (3) random forest classifier significantly outperforms a linear SVM. We built a demo application that maps two gestures to mouse position, click status, and wheel movement. We then used this mapping in the Google Earth application and were able to navigate with pan and zoom using just our gestures.

% Other things
We decided to make our implementation open source in hopes of attracting other developers to test and develop our application. We found this to be lacking with many other research projects in this area. They report the results, but do not make the code and testing data available.

Our experiments were done on four gestures, and the demo was with two gestures. We would like to train our system with more gestures and more diverse human hands to discover limitations of our system. For instance, such a system will be useful in detecting American Sign Langauge alphabet gestures.

\bibliographystyle{unsrt}
\begin{thebibliography}{9}

\bibitem{shotton2011} J. Shotton, A. Fitzgibbon, M. Cook, T. Sharp, M. Finocchio, R. Moore, A. Kipman, A. Blake. Real-time human pose recognition in parts from single depth images. CVPR, 2011.

\bibitem{wang2009} R. Wang and J. Popovi\'c. Real-time hand-tracking with a color glove. In Proc. ACM SIGGRAPH, 2009.

\bibitem{oikonomidis2011} I. Oikonomidis, N. Kyriazis, and A. Argyros. Efficient model-based 3D tracking of hand articulations using kinect. In BMVC, Aug 2011.

\bibitem{lepetit2005} V. Lepetit, P. Lagger, and P. Fua. Randomized trees for real-time keypoint recognition. In Proc. CVPR, pages 2:775-781, 2005. 

\bibitem{liblinear} R.-E. Fan, K.-W. Chang, C.-J. Hsieh, X.-R. Wang, and C.-J. Lin. LIBLINEAR: A library for large linear classification Journal of Machine Learning Research 9(2008), 1871-1874.

\bibitem{keim2002} D.A. Keim. Information visualization and visual data mining. Visualization and Computer Graphics, IEEE Transactions. Vol 8, no.1, pp. 1-8, Jan 2002.

\bibitem{stuerzlinger2010} W. Stuerzlinger, C Wingrave. The value of constraints for 3D user interfaces. Dagstuhl Seminar on VR, 2010.

\bibitem{hoffman2010} M. Hoffman, P. Varcholik, and J. LaViola. Breaking the status quo: improving 3D gesture recognition with spatially convenient input devices. IEEE VR, 2010.

\bibitem{alglib} V. Bystritsky. ALGLIB. 14 Aug 1999. Web. \url{http://www.alglib.net}.

\bibitem{googleearth} Google Inc. (2009). Google Earth (Version 6.2) [Software]. Available from \url{http://www.google.com/earth/index.html}.

\bibitem{mistry} Mistry, P., and Maes, P. (2009) SixthSense – A Wearable Gestural Interface. SIGGRAPH Asia ‘09, Emerging Technologies. Yokohama, Japan

\bibitem{minority} MIT Minority report

\end{thebibliography}

\end{document}

\section{Feature Extraction and Per-pixel Classification}
\label{sec: classification}

In this section, we describe the main classification algorithm used in the system. At each frame the system predicts each pixel as belonging to a gesture (e.g., open hand, close hand or background). The prediction result on every pixel are fed into a pooling algorithm to propose the final gesture location and type. The driving reason for us to choose per-pixel classification is that it allows massive parallelism using GPU. The prediction algorithm is identical for every pixel therefore we can leverage the Single Instruction Multiple Data architecture of the GPU.

\subsection{Feature Extractions}
\label{sec: feacture_extraction}
For each pixel \textbf{x}, we extract a set of features. Each feature corresponds to the depth difference between two  offset points, $\{\bf{u}, \bf{v} \}$,  from \textbf{x} and normalized by its depth. This is shown in equation~\ref{eqn:feature} where $d(x)$ is the depth at point $x$ and $\theta$ refers to the pair of offsets $u, v$.

\begin{align}
\label{eqn:feature}
f_{\theta}(x) = d\left(\textbf{x} + \frac{\textbf{u}}{d(\textbf{x})}\right) - d\left(\textbf{x} + \frac{\textbf{v}}{d(\textbf{x})}\right)
\end{align}

\begin{figure}
	\includegraphics[width=0.23\textwidth]{fig/OpenHandNearOffset.jpg}
	\includegraphics[width=0.23\textwidth]{fig/OpenHandFarOffset.jpg}
	
	\caption{Offset pairs for a given pixel. In the left figure, the reference pixel is the center of the palm. In the right figure, the reference pixel is the center of the palm of the person that stands. }
	\label{fig:offset}
\end{figure}

This feature extraction method is also used in \cite{shotton2011}. The offsets $\theta=\{\bf{u}, \bf{v} \}$ are generated randomly. In our design, we randomly sample the offsets from a bounded circular area. $\textbf{u}$ and $\textbf{v}$ are obtained as
\begin{align}
\label{enq: offest}
 (r \cos \beta, r \sin \beta),
\end{align}
where $r$ and $\beta$ are uniform random variables:
\begin{align}
 r &= U[R_{\text{min}}, R_{\text{min}}] \\
 \beta &= U[0, 2\pi]
\end{align}

As we can see from Figure \ref{fig:offset}, the offset pairs are located within the neighborhood of the palm and are proportional to the depth.

\subsection{The Classifier: SVM or Random Forest?}

In the training samples each pixel is labeled (we address how to generate massive labeled sample in Section \ref{sec: generating_training}) and we have to use supervised learning methods. Two methods come into our decision of choice: linear SVM and random forest. We use the following notation in our explanation -- the number of training samples: $l$, the number of features: $n$, the number of classes (types of gestures): $c$. Our main criteria for choosing the best algorithm are (1) prediction time and (2) accuracy.

\textbf{Linear SVM.} In linear SVM, a model that has $c$ classes and a weight vector of length $n$ is trained. The prediction is based on the dot product between the trained weight vectors and the feature vector. Therefore, the run-time complexity for predicting each pixel is $O(n\times c)$.

\textbf{Random Forest.} Random forest is built on an ensemble of decision trees that are trained on  bootstrap\footnote{Bootstrap means uniform   sampling with replacement} samples of the training sets. The decision tree in the random forest is a binary tree. Each node in the decision tree has one feature and a threshold to determine which branch to go to by comparing the feature value to the threshold. In the leafs of the tree are labels. The final prediction is made by the majority rule of the all decision trees in the forest. More details about the training and statistical analysis can be seen in \todo{Random Forest paper}. The depth of the decision tree is approximately as $O(\log l)$. Therefore $O(\log l)$ queries of features are made in a single decision tree when prediction. The run-time complexity for making prediction in random forest is then $O(\log l \times n_{\text{tree}})$. In fact, we can prune each decision tree to limit its depth. By pruning we mean once the node exceeds the maximum depth it becomes a leaf and the label would be the majority labels in the node's sub-tree. Pruning turns out to be crucial for us since it can guarantee the \textbf{worst-case} prediction time. Moreover pruning can make the decision tree almost balanced, ensuring the per-pixel prediction are done in almost constant time. Let us denote the maximum depth to be  $d_{\text{tree}}$, the run-time complexity can go down to $O(d_{\text{tree}}\times n_{\text{tree}})$. Finally, we present the random forest prediction algorithm in Algorithm \ref{alg: RF}.

\begin{algorithm}
%\vskip -0.1in
 \caption{Prediction Algorithm In Random Forest}
 \label{alg: RF}
\begin{algorithmic}

\State Input: a depth image and a pixel \textbf{x} to be predicted
\State Initialize predict\textunderscore list
   \For{each tree $i$ in the random forest}
     \State node $\leftarrow$ root of tree i
   	 \While{node is not a leaf}
	   \State feature\textunderscore index $\gets$ node.feature
       \State $(\textbf{u}, \textbf{v})$ $\gets$ get\textunderscore offset\textunderscore pair( \textbf{x}, feature\textunderscore index )
	   \State depth = get\textunderscore depth\textunderscore difference$(\textbf{u}, \textbf{v})$
	   \If{ depth $>$ node.threshold } 	
	      \State node $\gets$ node's right children	
	   \Else
	      \State node $\gets$ node's left children
       \EndIf
    \EndWhile
    \State predict\textunderscore list[i] $\gets$ node's label
  \EndFor
\\
\Return predict\textunderscore list's majority label
 
\end{algorithmic}
\end{algorithm}

\textbf{Comparison.} Let us do a simple calculation in which we use a realistic parameter setting. Suppose we have 2000 features, 3 classes, and using 3 trees with maximum depth of 20, the random forest is 100 times faster than linear SVM! Moreover, in our experimental evaluation, random forest has proven to be far superior than linear SVM in accuracy. Although linear SVM might use some advanced feature learning technique such as deep learning to achieve similar accuracy as to random forest, SVM is still too slow for us to adopt in the system. Inheriting from decision tree, random forest allows the system to extract features \textbf{on-demand}, which has been extremely crucial for real-time application as in the case of our system. The second author \todo{HOW ABOUT THE FIRST AUTHOR? He's never surprised} is really surprised by this analysis since he used to believe linear SVM is unbeatable in practice.




\textbf{Training.} Training in linear SVM can be very fast as it can has a run-time complexity of $O(n\times l)$ and there exists technique to scale the training to distributed system \todo{Cite Michael's paper}. Training in random forest, however, does not have an optimization-based foundation. We use brute force to determine the right feature and right threshold for each node in the decision forest. In determining the right threshold, we use grid search. In the training of random forest, it is highly recommended to put the training data in the main memory.


\subsubsection{GPU For Real-time Prediction}
There are $307,200$ $(640\times 480)$ pixels in a frame. Each pixel will undertake $O(d_{\text{tree}}\times n_{\text{tree}})$ operations for prediction. We first tried to implement the random forest prediction using CPU. It takes about 1 minute to process a frame! GPU has to be used since it allows massive parallelism. As can be seen in Figure~\ref{fig: GPUvsCPU}, GPU is more suitable for high-parallelism and high-latency jobs, while CPU is better fit for low-parallelism and low-latency jobs.

\begin{figure}
	\includegraphics[width=0.45 \textwidth]{fig/GPUvsCPU.pdf}
    \caption{A computational comparison between GPU and CPU. A simple experiment is conducted by varying the number of iterations in a for-loop. GPU would parallelize the for loop.}
    \label{fig: GPUvsCPU}
\end{figure}

In our system, each pixel has fired a thread to make prediction using random forest. In a typical GPU, more than thousands of threads are running concurrently thanks to the special architecture of GPU. As a result, per-pixel classification can be done in about 40ms \todo{Exact number}.

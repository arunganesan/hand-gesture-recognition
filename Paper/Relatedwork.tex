\section{Related Work \todo{Need more}}

There are two main techniques in hand gesture recognition - appearance based and model based. These are akin to probabilistic models of classification and generative models of classification respectively. \textbf{Appearance based} approaches read the pixels from the camera and build classifiers to label that as belonging to a finite set of classes. The main limitation of this technique is that the set of labels is finite and fixed ahead of time. The advantage of appearance based approaches is their implementation is often extremely fast and therefore suited for situations where live classification is important. Examples of appearance based techniques can be found in \cite{shotton2011, wang2009}. \textbf{Model based} approaches start with a set of hypotheses of the final classification based on rules of the object being classified. For instance, in the case of hand gestures a hypothesis can be a particular orientation of the joints.  An advantage here is that the hypotheses can be generated from an infinite space of possible classifications. The main disadvantage of model-based techniques is that they are often computationally expensive. In addition, model-based approaches tend to be very complex. An example of a model based technique can be found in \cite{oikonomidis2011}.
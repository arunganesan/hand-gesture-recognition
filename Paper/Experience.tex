\section{Experience}
\cutsection
\label{sec: experience}
We learned several lessons in building a practical machine learning system.

\cutequation
\textbf{Late-optimization.} Although our system is very time-sensitive, we found that it is not necessary to do optimization on one component while other components are not ready. We found it is best to develop quickly and do profiling to discover the bottleneck and optimize it. This saves us a lot of time in making unnecessary optimization.   

\cutequation
\textbf{Using pipeline rather than events.} In the testing of our system, we have incurred many concurrency bugs due to the event-driven programming framework. We solve this by not using locks but adding the processing unit to a pipeline, thus leaving us free concurrency bugs.

\cutequation
\textbf{Test accuracy is not enough.} We found that using the test accuracy is not a good evaluation of the actual system. Sometimes the system performs very well in the test data set, but poorly in practice. The reason behind this is that the environment we evaluate changes all the time: camera position, backgrounds, people's clothes, and etc. Therefore, we decide to choose some important parameters based on our experience in the actual environment instead of mere test accuracy. For example, although the experiments indicated the number of trees made no difference, we found to this to not be the case in actual tests with varying backgrounds. Therefore, we used a model trained with multiple trees rather than one.
\cutequation

\textbf{Bundle the model with feature extraction.} In our system, we separated feature extraction with the model. This turned out to be error-prone. It would happen sometimes that the trained model has the wrong feature extraction (e.g., the offset pairs are not correct), and things could get even worse when the real-time prediction system mistakenly uses the wrong feature extraction method. Our immediate solution is to be extremely careful to this. However, if we were to build the system again we would have  bundled the model with feature extraction to avoid human errors.

\cutsubsection
\subsection{Limitations}
\cutsubsection
Our current system is not without limitations. First, the real-time prediction component cannot achieve a frame rate of 30Hz but only 10Hz on average. Second, the random forest model takes a long time to train, especially with many training samples. Often times, training required around 24 hours or more. Third, the system must be retrained with more training samples every time the user wants to add a new gesture. 
